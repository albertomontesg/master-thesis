%!TEX root = ../thesis.tex

\begin{tikzpicture}
  \begin{axis}[
      set layers,
      width=\textwidth,
      height=.5\textwidth,
      width=\linewidth, % Scale the plot to \linewidth
      grid=both,
      grid style=dotted,
      minor ytick={0,0.05,...,1.1},
      ytick={0,0.1,...,1.1},
      yticklabels={0,.1,.2,.3,.4,.5,.6,.7,.8,.9,1},
      ymin=0, ymax=1,
      xmin=0, xmax=50,
      xlabel=Distance (pixels), % Set the labels
      ylabel=Detection Rate,
      legend columns=1,
      %transpose legend,
      legend pos= north west,
      legend style={/tikz/every even column/.append style={column sep=3mm}},
      % legend style={at={(0.5,-0.2)},anchor=north},
      % x tick label style={rotate=90,anchor=east}
    ]
    \addplot+[smooth,  mark=none, line width=1]
    table [x=Threshold, y=Baseline, col sep=comma] {data/precission_threshold_curves.csv};
    \addlegendentry{Baseline}

    \addplot+[smooth,  mark=none, line width=1]
    table [x=Threshold, y=Features ResNet Basic metriccosine patchsize3, col sep=comma] {data/precission_threshold_curves.csv};
    \addlegendentry{Pre-Trained VOC D=2048}

    \addplot+[smooth,  mark=none, line width=1]
      table [x=Threshold, y=Features Metric Learning (480p) Basic metriccosine patchsize3, col sep=comma] {data/precission_threshold_curves.csv};
    \addlegendentry{Triplet Loss D=2048}

    \addplot+[smooth,  mark=none, line width=1]
      table [x=Threshold, y=Features Metric Learning Triplet Loss (480p) D128 Basic metriccosine patchsize3, col sep=comma] {data/precission_threshold_curves.csv};
    \addlegendentry{Triplet Loss D=128}

    % \addplot+[smooth,  mark=none, line width=1]
    %   table [x=Threshold, y=Features Metric Learning Quadplet Loss (480p) Basic metriccosine patchsize3, col sep=comma] {data/precission_threshold_curves.csv};
    % \addlegendentry{Double Triplet Loss D=2048}

  \end{axis}
\end{tikzpicture}
