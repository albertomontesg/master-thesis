%!TEX root = ../thesis.tex

%% ----------------------------------------------------------------------------
% BIWI SA/MA thesis template
%
% Created 09/29/2006 by Andreas Ess
% Extended 13/02/2009 by Jan Lesniak - jlesniak@vision.ee.ethz.ch
%% ----------------------------------------------------------------------------
\newpage
\chapter{Experiments and Results}
% Describe the evaluation you did in a way, such that an independent researcher can repeat it. Cover the following questions:
% \begin{itemize}
%  \item \textit{What is the experimental setup and methodology?} Describe the setting of the experiments and give all the parameters in detail which you have used. Give a detailed account of how the experiment was conducted.
%  \item \textit{What are your results?} In this section, a \emph{clear description} of the results is given. If you produced lots of data, include only representative data here and put all results into the appendix.
% \end{itemize}

%%%%%%%%%%%%%%%%%%%%%%%%%%%%%%%%%

List all the experiments and its results regarding the tracking points on DAVIS\@.
Put mostly the results in the dashboard here.

Show the results of tracking points using triplet sampling. Improving the coverage.

Then the results pivoting to instance segmentation. Put here the results of the notebook.

\begin{figure}[h]
  \begin{center}
    \begin{tikzpicture}
      \begin{axis}[
          set layers,
          width=\textwidth,
          height=.5\textwidth,
          width=\linewidth, % Scale the plot to \linewidth
          grid=both,
          grid style=dotted,
          minor ytick={0,0.05,...,1.1},
          ytick={0,0.1,...,1.1},
          yticklabels={0,.1,.2,.3,.4,.5,.6,.7,.8,.9,1},
          ymin=0, ymax=1,
          xmin=0, xmax=50,
          xlabel=Threshold (pixels), % Set the labels
          ylabel=Precission,
          legend columns=1,
          %transpose legend,
          legend pos= south east,
          legend style={/tikz/every even column/.append style={column sep=3mm}},
          % legend style={at={(0.5,-0.2)},anchor=north},
          % x tick label style={rotate=90,anchor=east}
        ]
        \addplot+[smooth, index of colormap=1 of Set2-8, mark=none, line width=1]
        table [x=Threshold, y=Baseline, col sep=comma] {data/precission_threshold_curves.csv};
        \addlegendentry{Baseline}

        \addplot+[smooth, index of colormap=2 of Set2-8, mark=none, line width=1]
        table [x=Threshold, y=MDNet, col sep=comma] {data/precission_threshold_curves.csv};
        \addlegendentry{MDNet}

        \addplot+[smooth, index of colormap=3 of Set2-8, mark=none, line width=1]
        table [x=Threshold, y=Hourglass w/o Parent Network 1HG x2 sigma 5, col sep=comma] {data/precission_threshold_curves.csv};
        \addlegendentry{Hourglass}

        \addplot+[smooth, index of colormap=4 of Set2-8, mark=none, line width=1]
        table [x=Threshold, y=ResNet101 sigma20, col sep=comma] {data/precission_threshold_curves.csv};
        \addlegendentry{ResNet101}

        \addplot+[smooth, index of colormap=5 of Set2-8, mark=none, line width=1]
        table [x=Threshold, y=ResNet50 sigma40, col sep=comma] {data/precission_threshold_curves.csv};
        \addlegendentry{ResNet50}

        \addplot+[smooth, index of colormap=6 of Set2-8, mark=none, line width=1]
        table [x=Threshold, y=ResNet50 Multiscale sigma964824, col sep=comma] {data/precission_threshold_curves.csv};
        \addlegendentry{ResNet50 Multiscale}

      \end{axis}
    \end{tikzpicture}
    \caption{Precission performance.}
  \end{center}
\end{figure}

Tracking the features vector of the ResNet101 pretrained with VOC\@.
The best tracking was performed using cosine distance and averaging the representation of $3 \times 3$ patches.
Backbone architecture ResNet101.

\begin{figure}[h]
  \begin{center}
    \begin{tikzpicture}
      \begin{axis}[
          set layers,
          width=\textwidth,
          height=.5\textwidth,
          width=\linewidth, % Scale the plot to \linewidth
          grid=both,
          grid style=dotted,
          minor ytick={0,0.05,...,1.1},
          ytick={0,0.1,...,1.1},
          yticklabels={0,.1,.2,.3,.4,.5,.6,.7,.8,.9,1},
          ymin=0, ymax=1,
          xmin=0, xmax=50,
          xlabel=Threshold (pixels), % Set the labels
          ylabel=Precission,
          legend columns=1,
          %transpose legend,
          legend pos= south east,
          legend style={/tikz/every even column/.append style={column sep=3mm}},
          % legend style={at={(0.5,-0.2)},anchor=north},
          % x tick label style={rotate=90,anchor=east}
        ]
        \addplot+[smooth, index of colormap=1 of Set2-8, mark=none, line width=1]
        table [x=Threshold, y=Baseline, col sep=comma] {data/precission_threshold_curves.csv};
        \addlegendentry{Baseline}

        \addplot+[smooth, index of colormap=2 of Set2-8, mark=none, line width=1]
        table [x=Threshold, y=Features ResNet Basic metriccosine patchsize3, col sep=comma] {data/precission_threshold_curves.csv};
        \addlegendentry{Pre-Trained VOC D=2048}

        \addplot+[smooth, index of colormap=3 of Set2-8, mark=none, line width=1]
          table [x=Threshold, y=Features Metric Learning (480p) Basic metriccosine patchsize3, col sep=comma] {data/precission_threshold_curves.csv};
        \addlegendentry{Triplet Loss D=2048}

        \addplot+[smooth, index of colormap=4 of Set2-8, mark=none, line width=1]
          table [x=Threshold, y=Features Metric Learning Triplet Loss (480p) D128 Basic metriccosine patchsize3, col sep=comma] {data/precission_threshold_curves.csv};
        \addlegendentry{Triplet Loss D=128}

        \addplot+[smooth, index of colormap=5 of Set2-8, mark=none, line width=1]
          table [x=Threshold, y=Features Metric Learning Quadplet Loss (480p) Basic metriccosine patchsize3, col sep=comma] {data/precission_threshold_curves.csv};
        \addlegendentry{Double Triplet Loss D=2048}

        % \addplot+[smooth, index of colormap=2 of Set2-8, mark=none, line width=1]
        % table [x=Threshold, y=MDNet, col sep=comma] {data/precission_threshold_curves.csv};
        % \addlegendentry{MDNet}

        % \addplot+[smooth, index of colormap=3 of Set2-8, mark=none, line width=1]
        % table [x=Threshold, y=Hourglass w/o Parent Network 1HG x2 sigma 5, col sep=comma] {data/precission_threshold_curves.csv};
        % \addlegendentry{Hourglass}

        % \addplot+[smooth, index of colormap=4 of Set2-8, mark=none, line width=1]
        % table [x=Threshold, y=ResNet101 sigma20, col sep=comma] {data/precission_threshold_curves.csv};
        % \addlegendentry{ResNet101}

        % \addplot+[smooth, index of colormap=5 of Set2-8, mark=none, line width=1]
        % table [x=Threshold, y=ResNet50 sigma40, col sep=comma] {data/precission_threshold_curves.csv};
        % \addlegendentry{ResNet50}

        % \addplot+[smooth, index of colormap=6 of Set2-8, mark=none, line width=1]
        % table [x=Threshold, y="ResNet50 Multiscale sigma96,48,24", col sep=comma] {data/precission_threshold_curves.csv};
        % \addlegendentry{ResNet50 Multiscale}

      \end{axis}
    \end{tikzpicture}
    \caption{Precission performance for pixel representation tracking.}
  \end{center}
\end{figure}
