%!TEX root = ../thesis.tex

%% ----------------------------------------------------------------------------
% BIWI SA/MA thesis template
%
% Created 09/29/2006 by Andreas Ess
% Extended 13/02/2009 by Jan Lesniak - jlesniak@vision.ee.ethz.ch
%% ----------------------------------------------------------------------------


\chapter{Introduction}
Give an introduction to the topic you have worked on:

\begin{itemize}
 \item \textit{What is the rationale for your work?} Give a sufficient description of the problem, e.g. with a general description of the problem setting, narrowing down to the particular problem you have been working on in your thesis. Allow the reader to understand the problem setting.
 \item \textit{What is the scope of your work?} Given the above background, state briefly the focus of the work, what and how you did.
 \item \textit{How is your thesis organized?} It helps the reader to pick the interesting points by providing a small text or graph which outlines the organization of the thesis. The structure given in this document shows how the general structuring shall look like. However, you may fuse chapters or change their names according to the requirements of your thesis.
\end{itemize}


\section{Focus of this Work}

This thesis could be separated in two big scopes:

1. Tracking points through sequences using deep learning (everything related with training networks to predict the point)

2. Given a point and its representation, segment the object instance it belongs to.

\section{Thesis Organization}
