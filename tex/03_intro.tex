%!TEX root = ../thesis.tex

%% ----------------------------------------------------------------------------
% BIWI SA/MA thesis template
%
% Created 09/29/2006 by Andreas Ess
% Extended 13/02/2009 by Jan Lesniak - jlesniak@vision.ee.ethz.ch
%% ----------------------------------------------------------------------------


\chapter{Introduction}
Give an introduction to the topic you have worked on:

\begin{itemize}
 \item \textit{What is the rationale for your work?} Give a sufficient description of the problem, e.g. with a general description of the problem setting, narrowing down to the particular problem you have been working on in your thesis. Allow the reader to understand the problem setting.
 \item \textit{What is the scope of your work?} Given the above background, state briefly the focus of the work, what and how you did.
 \item \textit{How is your thesis organized?} It helps the reader to pick the interesting points by providing a small text or graph which outlines the organization of the thesis. The structure given in this document shows how the general structuring shall look like. However, you may fuse chapters or change their names according to the requirements of your thesis.
\end{itemize}


% \section{Focus of this Work}

% Which is the problem
The problem that this thesis have been trying to solve has been the segmentation of instances on images and videos.

% Who I tackled the problem
The solution proposed at the beggining consisted on focusing on video, try to track key points from instances.
The results weren't good, and that's why the we tried to tackle the problem using pixel embeddings.
With the pixel embeddings the instance segmentation on images was tackled.


% Aditional contributions I have made during my thesis
In adition to the research, I have contributed in side projects and research related with segmentation.
To help my research, and the others in the Segmentation Group at CVL, I developed a web interface which allowes to annotate and visualize results.
Also I have contributed on preparing the a new track on the segmentation field which consists on interactive segmentation. In relation with this, I helped collecting data with the annotation tool, and also develop a framework to evaluate interactive segmentation using human and automatic generated scribbles.



% \section{Thesis Organization}
