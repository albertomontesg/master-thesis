%!TEX root = ../thesis.tex

%\usepackage{url}
\usepackage{subfiles}
%\usepackage{setspace}
%\usepackage{multirow}

%Font 1
%\usepackage[T1]{fontenc}
%\usepackage{tgbonum}

%Font 2
\usepackage{fourier}
% \usepackage{mtpro2}

%Font 3
%\usepackage[T1]{fontenc}
%\usepackage{tgschola}

%\usepackage{kpfonts}

%\usepackage[table]{xcolor}
\usepackage[usenames,table]{xcolor}
\definecolor{webgreen}{rgb}{0, 0.7, 0} % less intense green
\definecolor{webblue}{rgb}{0, 0, 0.6} % less intense blue
\definecolor{webred}{rgb}{0.5, 0, 0} % less intense red

\usepackage[pagebackref=true,bookmarks=true, bookmarksnumbered=true, bookmarksopen=false, colorlinks=true,
    linkcolor= blue, urlcolor = webred, citecolor= webgreen, filecolor = webred]{hyperref}
\hypersetup{pdftitle={Master's thesis},pdfauthor={Alberto Montes G\'omez},
pdfkeywords={},bookmarksnumbered,pdfstartview={FitV}}%

\usepackage{algpseudocode}
\usepackage{algorithm}
%\usepackage{algorithm2e}

\usepackage{pdflscape}
\usepackage{multirow}

\usepackage{pgfplots}
\usepackage{tikz}
\usetikzlibrary{plotmarks}
\usetikzlibrary{patterns}
\usetikzlibrary{fit}
\pgfplotsset{compat=newest}
\pgfplotsset{every axis/.append style={
font=\footnotesize,
thin,
tick style={ultra thin}}}

\usetikzlibrary{trees,shapes.geometric,arrows,positioning}
% \usepackage[active,tightpage]{preview}
% \PreviewEnvironment{tikzpicture}

%\usepackage{marvosym}
\usepackage{amsmath} %,amssymb,amsthm}
%\usepackage[latin1]{inputenc}
\usepackage[english]{babel}

\usepackage[paperheight=297mm,paperwidth=210mm,top=35mm,left=36mm,height=235mm,width=148mm]{geometry}
%\usepackage{textcomp}
\usepackage{graphicx}


%%%%%%%%%%%%%%%%%%%%%%%%%%%%%%%%%%%%%%%%%%%%%%%%%%%%%%%%%%%%%%%%%%%%%%%%%%%%%%%%
% Commands
%%%%%%%%%%%%%%%%%%%%%%%%%%%%%%%%%%%%%%%%%%%%%%%%%%%%%%%%%%%%%%%%%%%%%%%%%%%%%%%%

% References
\newcommand{\figref}[1]{Figure~\ref{fig:#1}}
\newcommand{\equref}[1]{Equation~\ref{eq:#1}}
\newcommand{\tabref}[1]{Table~\ref{tab:#1}}

% Coloring
\definecolor{rowblue}{RGB}{220,230,240}

% Figures
\newcommand{\showpascalpca}[1]{%
  \begin{subfigure}{.33\textwidth}
    \includegraphics[width=1.\linewidth,height=.618\linewidth]{figures/pascal/image/#1.jpg}
  \end{subfigure}%
  \begin{subfigure}{.33\textwidth}
    \includegraphics[width=1.\linewidth,height=.618\linewidth]{figures/pascal/annotation/#1.png}
  \end{subfigure}%
  \begin{subfigure}{.33\textwidth}
    \includegraphics[width=1.\linewidth,height=.618\linewidth]{figures/pascal/pca/#1.jpg}
  \end{subfigure}
}

\newcommand{\showpascaldistancemap}[1]{%
  \begin{subfigure}{.33\textwidth}
    \includegraphics[width=1.\linewidth,height=.618\linewidth]{figures/pascal/image/#1.jpg}
  \end{subfigure}%
  \begin{subfigure}{.33\textwidth}
    \includegraphics[width=1.\linewidth,height=.618\linewidth]{figures/pascal/distance_map/#1_1.jpg}
  \end{subfigure}%
  \begin{subfigure}{.33\textwidth}
    \includegraphics[width=1.\linewidth,height=.618\linewidth]{figures/pascal/predicted_mask/#1_1.png}
  \end{subfigure}

  \begin{subfigure}{.33\textwidth}
    \includegraphics[width=1.\linewidth,height=.618\linewidth]{figures/pascal/annotation/#1.png}
  \end{subfigure}%
  \begin{subfigure}{.33\textwidth}
    \includegraphics[width=1.\linewidth,height=.618\linewidth]{figures/pascal/distance_map/#1_2.jpg}
  \end{subfigure}%
  \begin{subfigure}{.33\textwidth}
    \includegraphics[width=1.\linewidth,height=.618\linewidth]{figures/pascal/predicted_mask/#1_2.png}
  \end{subfigure}
}

\newcommand{\showdavisresults}[1]{%
  \begin{subfigure}{.2\textwidth}
    \includegraphics[width=1.\linewidth,height=.618\linewidth]{figures/davis/results/#1_1.jpg}
  \end{subfigure}%
  \begin{subfigure}{.2\textwidth}
    \includegraphics[width=1.\linewidth,height=.618\linewidth]{figures/davis/results/#1_2.jpg}
  \end{subfigure}%
  \begin{subfigure}{.2\textwidth}
    \includegraphics[width=1.\linewidth,height=.618\linewidth]{figures/davis/results/#1_3.jpg}
  \end{subfigure}%
  \begin{subfigure}{.2\textwidth}
    \includegraphics[width=1.\linewidth,height=.618\linewidth]{figures/davis/results/#1_4.jpg}
  \end{subfigure}%
  \begin{subfigure}{.2\textwidth}
    \includegraphics[width=1.\linewidth,height=.618\linewidth]{figures/davis/results/#1_5.jpg}
  \end{subfigure}
}


%%%%%%%%%%%%%%%%%%%%%%%%%%%%%%%%%%%%%%%%%%%%%%%%%%%%%%%%%%%%%%%%%%%%%%%%%%%%%%%%
% Legacy
%%%%%%%%%%%%%%%%%%%%%%%%%%%%%%%%%%%%%%%%%%%%%%%%%%%%%%%%%%%%%%%%%%%%%%%%%%%%%%%%


% To do heatmaps
\usepackage{collcell}

\newcommand*{\MinNumber}{0}%
\newcommand*{\MaxNumber}{1}%

\newcommand{\ApplyGradient}[1]{%
        \pgfmathsetmacro{\PercentColor}{100.0*(#1-\MinNumber)/(\MaxNumber-\MinNumber)}
        \hspace{-0.33em}\colorbox{blue!\PercentColor!white}{}
}

\newcolumntype{R}{>{\collectcell\ApplyGradient}c<{\endcollectcell}}




% \usepackage{algorithmic}
% \usepackage{algorithm}
% \usepackage[algo2e,ruled,lined]{algorithm2e}

%\usepackage{glossaries}
%\makeglossaries
%
%
%\newglossaryentry{uv}
%{name={$[u,v]$},
% description={Image pixel coordinates used in this work, where the origin is
%placed at the up-left corner of the image, $u$ is the horizontal direction,
%and $v$ the vertical one}
%}

\usepackage{booktabs}

\usepackage{etex}
\input xy
\xyoption{all}



\usepackage[compact]{titlesec}
%\titlespacing{\section}{0pt}{*0}{*0}
\titlespacing{\subsection}{0pt}{8pt}{0pt}
%\titlespacing{\subsubsection}{0pt}{*0}{*0}

%\usepackage[small,bf, justification=centering]{caption}
%\usepackage[format=hang]{caption}

\setlength{\parskip}{0.7\baselineskip}
\setlength{\parindent}{0pt}
\renewcommand{\baselinestretch}{1}

\usepackage{rotating}
% \usepackage{pifont}

\newcommand{\showglobaldisc}[1]{%
\mbox{\input{data/flat_meta_soa/global_#1.txt}\hspace{-2.5pt}}%
}
% \newcommand{\showswappedsoadisc}[1]{%
% \mbox{\input{data/flat_meta_soa/swapped_soa_disc_#1.txt}\hspace{-2.5pt}}%
% }
% \newcommand{\showswappedhumdisc}[1]{%
% \mbox{\input{data/flat_meta_soa/swapped_human_disc_#1.txt}\hspace{-2.5pt}}%
% }
% \newcommand{\showbaselinedisc}[1]{%
% \mbox{\input{data/flat_meta_soa/baseline_disc_#1.txt}\hspace{-2.5pt}}%
% }

% \newcommand{\showodsf}[2]{%
% \mbox{\input{data/flat_pr_curves/#1_#2_ods_f.txt}\hspace{-2.5pt}}%
% }
% \newcommand{\showodsfb}[2]{%
% \mbox{\input{data/segm_bsds/test_#1_#2_ods_f.txt}\hspace{-2.5pt}}%
% }
% \newenvironment{llista}{\begin{list}{$\bullet$}{\leftmargin=1em\partopsep=0pt\topsep=0pt\itemsep=0pt}}{\end{list}}

% \newenvironment{highlight}{\vspace{5pt}
% \begin{tabular}{@{\hspace{-1.5mm}}c|@{\hspace{4mm}}r}
% &\begin{minipage}{0.95\linewidth}
% }{%
% \end{minipage}
% \end{tabular}
% \vspace{5pt}
% }

\usepackage{paralist}

%********Capcalera i peu de pagina**********
\pagestyle{myheadings}
\renewcommand{\chaptermark}[1]{\markboth{#1}{}}
\renewcommand{\sectionmark}[1]{\markright{\thesection\ #1}}

% \newcommand{\showone}[1]{%
% \mbox{\fbox{\includegraphics[width=0.23\linewidth]{figures/obj_cands/PASCAL_1_#1_im.jpg}}
% \hspace{2mm}
% \fbox{\includegraphics[width=0.23\linewidth]{figures/obj_cands/PASCAL_1_#1_ucm.png}}
% \hspace{2mm}
% \fbox{\includegraphics[width=0.23\linewidth]{figures/obj_cands/PASCAL_1_#1_gt.png}}
% \hspace{2mm}
% \fbox{\includegraphics[width=0.23\linewidth]{figures/obj_cands/PASCAL_1_#1_mask_1.png}}}%
% }

% \newcommand{\showtwo}[1]{%
% \mbox{\fbox{\includegraphics[width=0.18\linewidth]{figures/obj_cands/PASCAL_2_#1_im.jpg}}
% \hspace{1.5mm}
% \fbox{\includegraphics[width=0.18\linewidth]{figures/obj_cands/PASCAL_2_#1_ucm.png}}
% \hspace{1.5mm}
% \fbox{\includegraphics[width=0.18\linewidth]{figures/obj_cands/PASCAL_2_#1_gt.png}}
% \hspace{1.5mm}
% \fbox{\includegraphics[width=0.18\linewidth]{figures/obj_cands/PASCAL_2_#1_mask_1.png}}
% \hspace{1.5mm}
% \fbox{\includegraphics[width=0.18\linewidth]{figures/obj_cands/PASCAL_2_#1_mask_2.png}}}%
% }

% \newcommand{\showthree}[1]{%
% \mbox{\fbox{\includegraphics[width=0.15\linewidth]{figures/obj_cands/PASCAL_3_#1_im.jpg}}
% \hspace{1.5mm}
% \fbox{\includegraphics[width=0.15\linewidth]{figures/obj_cands/PASCAL_3_#1_ucm.png}}
% \hspace{1.5mm}
% \fbox{\includegraphics[width=0.15\linewidth]{figures/obj_cands/PASCAL_3_#1_gt.png}}
% \hspace{1.5mm}
% \fbox{\includegraphics[width=0.15\linewidth]{figures/obj_cands/PASCAL_3_#1_mask_1.png}}
% \hspace{1.5mm}
% \fbox{\includegraphics[width=0.15\linewidth]{figures/obj_cands/PASCAL_3_#1_mask_2.png}}%
% \hspace{1.5mm}
% \fbox{\includegraphics[width=0.15\linewidth]{figures/obj_cands/PASCAL_3_#1_mask_3.png}}}%
% }

% \newcommand{\headerone}{\noindent\mbox{\begin{minipage}{0.23\linewidth}\centering Original Image\end{minipage}
% \hspace{2mm}
% \begin{minipage}{0.23\linewidth}\centering Multiscale hier.\end{minipage}
% \hspace{2mm}
% \begin{minipage}{0.23\linewidth}\centering Ground truth\end{minipage}
% \hspace{2mm}
% \begin{minipage}{0.23\linewidth}\centering MCG best cand. among 400\end{minipage}}\\[2mm]}


% \newcommand{\headertwo}{\noindent\mbox{\begin{minipage}{0.18\linewidth}\centering Original Image\end{minipage}
% \hspace{1.5mm}
% \begin{minipage}{0.18\linewidth}\centering Multiscale hier.\end{minipage}
% \hspace{1.5mm}
% \begin{minipage}{0.18\linewidth}\centering Ground truth\end{minipage}
% \hspace{1.5mm}
% \begin{minipage}{0.39\linewidth}\centering MCG best candidates among 400\end{minipage}}\\[2mm]}


% \newcommand{\headerthree}{\noindent\mbox{\begin{minipage}{0.15\linewidth}\centering Original Image\end{minipage}
% \hspace{1.5mm}
% \begin{minipage}{0.15\linewidth}\centering Multiscale hier.\end{minipage}
% \hspace{1.5mm}
% \begin{minipage}{0.15\linewidth}\centering Ground truth\end{minipage}
% \hspace{1.5mm}
% \begin{minipage}{0.48\linewidth}\centering MCG best candidates among 400\end{minipage}}\\[2mm]}



% \pgfplotsset{
% cycle list={%
% {red, mark=*, mark size=1.3},
% %{red, mark=diamond*, mark size=1.5},
% {red, mark=square*, mark size=1.25},
% {red, mark=triangle*, mark size=1.6},
% {blue,mark=o, mark size=1.3},
% %{blue,mark=diamond, mark size=1.5},
% {blue,mark=square, mark size=1.25},
% {blue,mark=triangle, mark size=1.6},
% }}
\newtheorem{example}{Example}



%% Change chapter style
\usepackage{calc}
\definecolor{chaptercolor}{gray}{0.8}
% helper macros
\newcommand\numlifter[1]{\raisebox{-2cm}[0pt][0pt]{\smash{#1}}}
\newcommand\numindent{\kern37pt}
\newlength\chaptertitleboxheight
\makechapterstyle{hansen}{
  \renewcommand\printchaptername{\raggedleft}
  \renewcommand\printchapternum{%
    \begingroup%
    \leavevmode%
    \chapnumfont%
    \strut%
    \numlifter{\thechapter}%
    \numindent%
\endgroup%
}
  \renewcommand*{\printchapternonum}{%
    \vphantom{\begingroup%
      \leavevmode%
      \chapnumfont%
      \numlifter{\vphantom{9}}%
      \numindent%
      \endgroup}
    \afterchapternum}
  \setlength\midchapskip{0pt}
  \setlength\beforechapskip{0.5\baselineskip}
  \setlength{\afterchapskip}{3\baselineskip}
  \renewcommand\chapnumfont{%
    \fontsize{4cm}{5mm}%
    \bfseries%
    %\sffamily%
    \color{chaptercolor}%
  }
  \renewcommand\chaptitlefont{%
    \normalfont%
    \huge%
    \bfseries%
    \raggedleft%
  }%
  \settototalheight\chaptertitleboxheight{%
    \parbox{\textwidth}{\chaptitlefont \strut bg\\bg\strut}}
  \renewcommand\printchaptertitle[1]{%
    \parbox[t][\chaptertitleboxheight][t]{\textwidth}{%
      %\microtypesetup{protrusion=false}% add this if you use microtype
      \chaptitlefont\strut ##1\strut}%
}}
\chapterstyle{hansen}
\aliaspagestyle{chapter}{empty}


\usepackage{caption}
\usepackage{subcaption}
\newcommand{\subcaptationspace}{\captionsetup{skip=2.5pt} }
\newcommand{\captationspace}{\captionsetup{skip=2.5pt} }
\newcommand{\subfigurespace}{\par\medskip}%\bigskip or \medskip

% Additions for the DAVIS paper
% \definecolor{rowblue}{RGB}{220,230,240}
% \renewcommand{\bf}{\textbf}
% \newcommand{\namemcg}{MCG}
% \newcommand{\namesfrgb}{SF-LAB}
% \newcommand{\namesfflow}{SF-MOT}
% \newcommand{\namenonlocalconsensus}{NLC}
% \newcommand{\namecvos}{CVOS}
% \newcommand{\nametracingdisco}{TRC}
% \newcommand{\namemoseg}{MSG}
% \newcommand{\namekeysegments}{KEY}
% \newcommand{\namesaliencyseg}{SAL}
% \newcommand{\namefastvideoseg}{FST}
% \newcommand{\nametsp}{TSP}
% \newcommand{\nameseamseg}{SEA}
% \newcommand{\namehvs}{HVS}
% \newcommand{\namejumpcut}{JMP}
% \newcommand{\namefcop}{FCP}
% \newcommand{\cmark}{\checkmark}
